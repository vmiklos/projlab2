\documentclass[hyperref={pdfpagelabels=false}]{beamer}
\author{Vajna Mikl�s}

\setbeamertemplate{background canvas}[vertical shading][bottom=white,top=structure.fg!25]

\usetheme{Warsaw}
\setbeamertemplate{headline}{}
\setbeamertemplate{footline}[page number]
\setbeamersize{text margin left=0.5cm}
  
\usepackage[magyar, english]{babel}

\usepackage{times}
\usepackage[latin2]{inputenc}
\usepackage[T1]{fontenc}

\begin{document}

\title{A dm-mirror hibat�r�se �s teljes�tm�nye}
\date{2010. december 9.}

\frame{\titlepage}

\begin{frame}
\frametitle{Bevezet�}
%\begin{figure}[H]
%\includegraphics[width=47mm,keepaspectratio]{model.eps}
%\end{figure}
\begin{itemize}
\item A dm-mirror egy tipikus 3 r�teg� architekt�ra legals� szintje \emph{alatt} helyezkedik el
\item Egy szoftveres t�kr�z�si megold�s a Linux kernelben
\item Egyed�l�ll� tulajdons�ga, hogy a device-mapper keretrendszerre �p�l (logikai k�tetkezel�s)
\item Probl�ma: szuboptim�lis teljes�tm�ny ha a fizikai diszkek el�r�si
sebess�ge nem azonos
\end{itemize}
\end{frame}

\end{document}
